\documentclass[english]{article}
\usepackage[latin1]{inputenc}
\usepackage{babel}
\usepackage{verbatim}

%% do we have the `hyperref package?
\IfFileExists{hyperref.sty}{
   \usepackage[bookmarksopen,bookmarksnumbered]{hyperref}
}{}

%% do we have the `fancyhdr' package?
\IfFileExists{fancyhdr.sty}{
\usepackage[fancyhdr]{latex2man}
}{
%% do we have the `fancyheadings' package?
\IfFileExists{fancyheadings.sty}{
\usepackage[fancy]{latex2man}
}{
\usepackage[nofancy]{latex2man}
\message{no fancyhdr or fancyheadings package present, discard it}
}}

\setVersionWord{Version:}  %%% that's the default, no need to set it.
\setVersion{0.0.1}

\begin{document}

\begin{Name}{1}{mp\_suggest}{Elf M. Sternberg}{Utilities}{mp\_suggest \\--\\ MP3 Encoding Suggestions}

  \Prog{mp\_suggest} is a tool to compare your existing MP3 files to
  their ID3 entries and suggest possible new entries.

\end{Name}

\section{Synopsis}
%%%%%%%%%%%%%%%%%%

\Prog{mp\_suggest} \oOptArg{-g}{genre}
                 \oOptArg{-r}{artist}
                 \oOptArg{-a}{album}
                 \oOpt{-d}
                 \oOpt{-n}
                 \oOpt{-h}
                 \oOpt{-v}

\section{Description}
%%%%%%%%%%%%%%%%%%%%%

\Prog{mp\_suggest} works on a directory of MP3 files and attempts to
determine the album name, artist name, genre, title, and album order
from various clue: the content of ID3 files, the sort order, and the
name of the directory.  It assumes that your MP3 filenames vaguely
resemble the names of the songs, that they're organized with track
number prefixes, and ideally that the directory is in ``Artist Name -
Album Name'' format.

It then generates a shell script suitable to piping into bash to change
the details of all the MP3 files to match your specifications.  The
commands can be tweaked, and all of them override the techniques
\Prog{mp\_suggest} uses to derive information.

\section{Requirements}
%%%%%%%%%%%%%%%%%%%%%%

\begin{description}\setlength{\itemsep}{1cm}
\item[Hy] \Prog{mp\_suggest} requires Hy version $>=$ 0.10.0, as well as
  a local install of both eyeD3 and django.  I'll try to get these
  fixed eventually.

\end{description}

\section{Version}
%%%%%%%%%%%%%%%%%
Version: 0.0.1 of December 6, 2014

\section{License and Copyright}
%%%%%%%%%%%%%%%%%%%%%%%%%%%%%%%

\begin{description}
\item[Copyright] \copyright\ 2001, Elf M. Sternberg \\
        \Email{elf.sternberg@gmail.com}  \\
        \URL{http://elfsternberg.com}

\item[License] This program can be redistributed and/or modified under the
  terms of the GNU Public License.  See the file COPYING or visit
  \URL{http://www.gnu.org/copyleft/gpl.html} for terms of use and
  distribution.

\item[Misc] If you find this software useful, please drop me some email
\end{description}

\section{Author}
%%%%%%%%%%%%%%%%

\noindent
Elf M. Sternberg                      \\
Email: \Email{elf.sternberg@gmail.com}        \\
WWW: \URL{http://elfsternberg.com}.
\LatexManEnd
\end{document}
