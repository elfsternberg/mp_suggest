\documentclass[english]{article}
\usepackage[latin1]{inputenc}
\usepackage{babel}

\IfFileExists{fancyhdr.sty}{
\usepackage[fancyhdr,pdf]{latex2man}
}{

\IfFileExists{fancyheadings.sty}{
\usepackage[fancy,pdf]{latex2man}
}{
\usepackage[nofancy,pdf]{latex2man}
\message{no fancyhdr or fancyheadings package present, discard it}
}}

\setVersionWord{Version:}
\setVersion{0.1a}

\begin{document}

\begin{Name}{1}{mp\_suggest}{Elf M. Sternberg}{Utilities}{mp\_suggest \\--\\ MP3 Encoding Suggestions}

  \Prog{mp\_suggest} is a tool to compare your existing MP3 files to
  their ID3 entries and suggest possible new entries.

\end{Name}

\section{Synopsis}
%%%%%%%%%%%%%%%%%%

\Prog{fcc} \OptArg{-class }{list of class names}
                \oOpt{-overwrite}
                \oOptArg{-settings_file }{filename}

\section{Description}
%%%%%%%%%%%%%%%%%%%%%
\Prog{fcc} takes the name of one or more classes and generates the
  appropriate code and header files for those classes.  By default, it
  includes a constructor, destructor, private copy constructor, private
  assignment operator, unit testing, and appropriate inlines.  It also
  generates a raw BSD-style makefile.

\section{Options}
%%%%%%%%%%%%%%%%%
\begin{Description}
\item[\OptArg{-class }{list of class names}] Names of classes to be
  generated.  The files will be named after the first class name on the
  list.  Class names must be separated by whitespace.

\item[\OptArg{-author }{author's name}] Your name here.

\item[\OptArg{-namespace }{namespace name}] Wrap the output in a
  specified namespace.

\item[\Opt{-verbose}] Explain in excruciating detail what the program
  is doing.

\item[\Opt{-no_unit_test}] Do not generate unit-testing definitions in
  the source files.  This is an empty method; it's up to the user to
  define appropriate tests.

\item[\Opt{-no_makefile}] Do not generate a makefile for this code.

\item[\OptArg{-project }{project_name}] Define a project for this makefile.  Right now
  this option does nothing.

\item[\Opt{-sccs_keywords}] Generate IDs appropriate to SCCS.  

\item[\Opt{-continuus_keywords}] Generate keywords for the Continuus
  Version System.

\item[\Opt{-open_source_notice}] Add some comments about how this code
  is open source and permission to copy in granted yadda yadda...

\item[\OptoArg{-copyright }{copyright}] Adds your copyright notice to the
  code.  

\item[\OptArg{-base_filename }{base_filename}] Set the name of the
  files output.  Defaults to first class name listed if not set.

\item[\Opt{-no_copy_ctor}] Do not provide a copy constructor in the
  output code.

\item[\Opt{-no_assignment_operator}] Do not provide an assignment
  operator in the output code.

\item[\Opt{-no_ctor}] Do not provide a default constructor in the
  output code.

\item[\Opt{-no_dtor}] Do not provide a destructor in the output code.

\item[\Opt{-public_copy_ctor}] Put the copy constructor in the public
  section instead of the private section.

\item[\Opt{-public_assignment_operator}] Put the assignment operator in
  the public section instead of the private section.

\item[\Opt{-no_dump_diagnostics}] Do not provide a debugging method.

\item[\Opt{-no_check_valid}] Do not provide a Programming By Contract
  style method to check for invariants.

\item[\Opt{-no_icc}] Do not provide a file for class inlines.

\item[\OptArg{-settings_file }{filename}] Get options from a settings
  file.  Right now this option does nothing.

\section{Requirements}
%%%%%%%%%%%%%%%%%%%%%%

\begin{description}\setlength{\itemsep}{1cm}
\item[Python] \Prog{fcc} requires Python version $>=$ 2.0
\end{description}

\section{Version}
%%%%%%%%%%%%%%%%%
Version: 0.8a of May 5, 2001

\section{License and Copyright}
%%%%%%%%%%%%%%%%%%%%%%%%%%%%%%%

\begin{description}
\item[Copyright] \copyright\ 2001, Elf M. Sternberg \\
        \Email{elf.sternberg@gmail.com}  \\
        \URL{http://elfsternberg.com}

\item[License] This program can be redistributed and/or modified under the
  terms of the GNU Public License.  See the file COPYING or visit
  \URL{http://www.gnu.org/copyleft/gpl.html} for terms of use and
  distribution.

\item[Misc] If you find this software useful, please drop me some email

\item[Misc] This man page was generated using latex2man by Dr. Jurgen
Vollmer, and is available from the CTAN archives.
\end{description}

\section{Author}
%%%%%%%%%%%%%%%%

\noindent
Elf M. Sternberg                      \\
Email: \Email{elf.sternberg@gmail.com}        \\
WWW: \URL{http://elfsternberg.com}.
\LatexManEnd
\end{document}
